%!TEX root=main.tex
\section{Efficient NTT and Convolution} \label{section:ntt}
Now that we have a large enough principal root of unity, and a way to pad inputs to arbitrary greater lengths, it seems like we just need to plug in an off-the-shelf NTT/FFT algorithm and finish our job. Unfortunately, the very limited roots of unity in our setting severely limit the scope of applicable algorithms.

\begin{example}
    Let \(p = 2\). \Cref{prop:LiftedPrimPoly} guarantees principal roots of unity of order \(N \mid 2^r - 1\). But \(N\) is then always odd, hence the famous Cooley-Tuckey radix 2 algorithm~\cite{IEEE:Cooley67}, split radix 4 algorithm~\cite{EL:Duhamel84}, as well as Bluestein FFT algorithm~\cite{IEEE:Bluestein70} cannot be applied here.
\end{example}
\begin{example}
    Let \(p =2\) and \(N = 1000\). We could instead use a radix \(q\) Cooley-Tuckey algorithm, for instance \(q = 3\). The smallest power of \(3 \ge 2N - 1\) is \(M = 3^7 = 2187\), and the smallest \(r\) such that \(M \mid 2^r - 1\) is the multiplicative order of \(2 \bmod 3^7\) and equals \(r = 1458\). The huge polynomial degree clearly renders the computation impractical.
\end{example}

These 2 examples suggest that a pure-radix strategy is not generally applicable or practical, and therefore we need to incorporate mixed radix strategy. Our method is the combination of the following 2 algorithms.
\begin{enumerate}
    \item The first algorithm handles the case of prime power length. Here we depart from FFT methodology and consider another \(O(N \log N)\) algorithm that computes the circular convolution in 1 pass. Our algorithm is a slightly generalized version found in \cite{ARXIV:Rosowski21} that handles arbitrary radix.
    \item The second algorithm is essentially the Good-Thomas Prime Factorization algorithm~\cite{JRSS:Good58,ADC:Thomas63}. If the length \(N = N_1 N_2\) can be factored into coprime factors, this method reduces the computation of length \(N\) fourier transform to computations of block-wise fourier transforms of length \(N_1\) and \(N_2\)
\end{enumerate}

In \cref{subsection:combination} the 2 algorithms above are combined to handle smooth convolution length \(N = q_1 q_2 \ldots q_{n-1} q_n^e\) where \(q_1, \ldots, q_n\) are small primes and \(e \ge 1\).

\subsection{Prime Power Case} \label{subsection:primePower}
First we recall the notion of an \(f\)-circulant matrix.
\begin{definition}{\(f\)-circulant matrix} \label{def:CircMat}
    Let \(\expdVec{a}{N}\) be a sequence and \(f\) a number. The \(f\)-circulant matrix associated with \(\expdVec{a}{N}\) is an \(N \times N\) matrix \(\bm{A}\) satisfying:
    \[\forall 0 \le i,j < N:\: \bm{A}_{i,j} = \begin{cases}
        a_{j - i} & \text{if \(i \le j\)} \\
        f a_{N+j-i} & \text{otherwise}
    \end{cases}\]
\end{definition}
The \(1\)-circulant matrix is just a circulant matrix, while a \((-1)\)-circulant matrix is usually called a nega-cyclic circulant matrix.

\ifFullVersion
We first present the simple and elegant method described in \cite{ARXIV:Rosowski21}
\begin{theorem}{\cite{ARXIV:Rosowski21}} \label{thm:CircMult1}
    Let \(\RingR\) be a commutative ring with identity, \(N\) a power of \(2\) and \(f \in \RingR\). Suppose
    \begin{itemize}
        \item \(\RingR\) contains an \(N\textsuperscript{th}\) root of unity and an \(N\textsuperscript{th}\) root of \(f\)
        \item \(2, f\) both have multiplicative inverse in \(\RingR\)
    \end{itemize}
    Then the multiplication of an \(f\)-circulant matrix \(\bm{A}\in \RingR^{N \times N}\) and a vector \(\bm{b} \in \RingR^N\) can be done with \(O(N \log N)\) ring operations using \cref{alg:Mult1}.
\end{theorem}
\begin{algorithm}[h]
    \caption{CircMatMult: Multiply an \(f\)-circulant matrix with a vector}\label{alg:Mult1}
    \begin{algorithmic}[1]
        \Require \(f\)-circulant matrix \(\bm{A}\), input vector \(\bm{b}\), \(f\) and length \(N\)
        \Ensure product \(\bm{A} \cdot \bm{b}\)
        \If {\(N = 2\)} \Comment{Base Case}
            \State {\bf Return} \(\bm{A} \cdot \bm{b}\)
        \EndIf
        \State Decompose \(\bm{A} = \begin{pmatrix}
            \bm{A}_1 & \bm{A}_2 \\ f \bm{A}_2 & \bm{A}_1
        \end{pmatrix}\) where \(\bm{A_1}, \bm{A_2} \in \RingR^{N/2 \times N/2}\). Parse \(\bm{b} = (\bm{b}_1 \parallel \bm{b}_2)^{\top}\) where \(\bm{b}_1, \bm{b}_2 \in \RingR^{N/2}\)
        \State Calculate
            \begin{align*}
                \bm{M}_1 &= \bm{A}_1 + \sqrt{f} \bm{A}_2 \in \RingR^{N/2 \times N/2} \\
                \bm{M}_2 &= \bm{A}_1 - \sqrt{f} \bm{A}_2 \in \RingR^{N/2 \times N/2} \\
                \bm{d}_1 &= \sqrt{f} \bm{b}_1 + \bm{b}_2 \in \RingR^{N/2} \\
                \bm{d}_2 &= \sqrt{f} \bm{b}_1 - \bm{b}_2 \in \RingR^{N/2}
            \end{align*}
        \State Recursively calculate
            \begin{align*}
                \bm{e}_1 &= \text{CircMatMult}(\bm{M}_1, \bm{d}_1, \sqrt{f}, \frac{N}{2}) \\
                \bm{e}_2 &= \text{CircMatMult}(\bm{M}_2, \bm{d}_2, -\sqrt{f}, \frac{N}{2})
            \end{align*}
        \State Compute \(\bm{c}_1 = \frac{\bm{e}_1 + \bm{e}_2}{2\sqrt{f}}\), \(\bm{c}_2 = \frac{\bm{e}_1 + \bm{e}_2}{2} - \bm{e}_2\)
        \State {\bf Return} \(\bm{c} = (\bm{c}_1 \parallel \bm{c}_2)^{\top} \)
    \end{algorithmic}
\end{algorithm}
\begin{proof}
    On one hand, \(\begin{pmatrix}
        \bm{A}_1 & \bm{A}_2 \\ f \bm{A}_2 & \bm{A}_1
    \end{pmatrix} \begin{pmatrix}
        \bm{b}_1 \\ \bm{b}_2
    \end{pmatrix} = \begin{pmatrix}
        \bm{A}_1 \bm{b}_1 + \bm{A}_2 \bm{b}_2 \\
        \bm{A}_1 \bm{b}_2 + f \bm{A}_2 \bm{b}_1
    \end{pmatrix}\)

    Assume the recursion step of the algorithm gives the correct result. Then
    \begin{align*}
        \bm{e}_1 = \bm{M}_1 \bm{d}_1 &= \sqrt{f}(\bm{A}_1 \bm{b}_1 + \bm{A}_2 \bm{b}_2) + (\bm{A}_1 \bm{b}_2 + f \bm{A}_2 \bm{b}_1) \\
        \bm{e}_2 = \bm{M}_2 \bm{d}_2 &= \sqrt{f}(\bm{A}_1 \bm{b}_1 + \bm{A}_2 \bm{b}_2) - (\bm{A}_1 \bm{b}_2 + f \bm{A}_2 \bm{b}_1)
    \end{align*}
    Hence the final result is correct.

    Next we claim that \(\bm{M}_1, \bm{M_2}\) are respectively \(\sqrt{f}, -\sqrt{f}\)-circulant matrices. This ensures that the recursion step is correct. Below we prove this fact for \(\bm{M}_1\).

    Since \(\bm{A}\) is \(f\)-circulant, \(\exists (a)_{i=0}^{N-1}\) such that \(\bm{A}_{i,j} = \begin{cases}
        a_{j-i} & \text{if \(i \le j\)} \\
        f a_{N+j-i} & \text{otherwise}
    \end{cases}\). 
    
    For \(0 \le i < \frac{N}{2}\), define \(\alpha_i = a_i + \sqrt{f}a_{\frac{N}{2} + i}\).

    If \(0 \le i \le j < \frac{N}{2}\):
    \begin{align*}
        (\bm{M}_1)_{i,j} &= (\bm{A}_1)_{i,j} + \sqrt{f}(\bm{A}_2)_{i,j} = \bm{A}_{i,j} + \sqrt{f} \bm{A}_{i, j + \frac{N}{2}} \\
        &= a_{j-i} + \sqrt{f} a_{j + \frac{N}{2} - i} = \alpha_{j-i}
    \end{align*}

    If \(0 \le j < i < \frac{N}{2}\):
    \begin{align*}
        (\bm{M}_1)_{i,j} &= \bm{A}_{i,j} + \sqrt{f} \bm{A}_{i, j + \frac{N}{2}} = f a_{N+j-i} + \sqrt{f}a_{j + \frac{N}{2} - i} \\
        &= \sqrt{f}(a_{\frac{N}{2}+j-i} + \sqrt{f}a_{\frac{N}{2} + \frac{N}{2} + j - i}) = \sqrt{f} \alpha_{\frac{N}{2}+j-i}
    \end{align*}

    Hence \(\bm{M}_1\) is \(\sqrt{f}\) circulant. The case of \(\bm{M}_2\) follows similarly.

    Finally, the algorithm follows the classical divide and conquer qpproach and hence has time complexity \(O(N \log N)\).
\end{proof}
\fi

We now propose a generalized version of the algorithm described in \cite{ARXIV:Rosowski21}. Ours support any radix \(B\) that satisfies the condition in the following theorem.
\begin{theorem} \label{thm:CircMult2}
    Let \(\RingR\) be a commutative ring with identity, \(B > 0\) and \(N\) a power of \(B\). Let \(f \in \RingR\). Suppose
    \begin{itemize}
        \item \(\RingR\) contains an \(N\textsuperscript{th}\) root of unity and an \(N\textsuperscript{th}\) root of \(f\)
        \item \(B, f\) both have multiplicative inverse in \(\RingR\)
    \end{itemize}
    Then the multiplication of an \(f\)-circulant matrix \(\bm{A}\in \RingR^{N \times N}\) and a vector \(\bm{b} \in \RingR^N\) can be done with \(O(N \log N)\) ring operations using \cref{alg:Mult2}.
\end{theorem}

\begin{algorithm}[ht]
    \caption{GenCircMatMult: Multiply an \(f\)-circulant matrix with a vector}\label{alg:Mult2}
    \begin{algorithmic}[1]
        \Require \(f\)-circulant matrix \(\bm{A}\), input vector \(\bm{b}\), \(f\) and length \(N\)
        \Ensure product \(\bm{A} \cdot \bm{b}\)
        \If {\(N = B\)} \Comment{Base Case}
            \State {\bf Return} \(\bm{A} \cdot \bm{b}\)
        \EndIf
        \State Decompose \(\bm{A} = \begin{pmatrix}
            \bm{A}_0 & \bm{A}_1 & \dots & \bm{A}_{B-1} \\ 
            f \bm{A}_{B-1} & \bm{A}_0 & \dots &\bm{A}_{B-2} \\
            & & \vdots & \\
            f \bm{A}_1 & f \bm{A}_2 & \dots & \bm{A}_0
        \end{pmatrix}\) where \(\bm{A_i}\in \RingR^{N/B \times N/B}\). Parse \(\bm{b} = \expdVecc{\bm{b}}{B}^{\top}\) where \(\bm{b}_i \in \RingR^{N/B}\)
        \State Let \(\omega\) be a primitive \(B\textsuperscript{th}\) root of unity, \(r = f^{1/B}\) a \(B\textsuperscript{th}\) root of \(f\)
        \For {\(i \gets 0 \text{ to } B-1\)}
            \State Compute
            \[
                \bm{M}_i = \sum_{j=0}^{B-1}r^j \omega^{ij} \bm{A}_j, \qquad
                \bm{d}_i = \sum_{j=0}^{B-1}r^{B-1-j} \omega^{-ij} \bm{b}_j
            \]
            \State Recursively compute \(\bm{e}_i = \text{GenCircMatMult}(\bm{M}_i, \bm{d}_i, r \omega^i, \frac{N}{B})\)
        \EndFor
        \For {\(i \gets 0 \text{ to } B-1\)}
            \State Calculate \(\bm{c}_i = \left(B r^{B-1-i}\right)^{-1} \sum_{j=0}^{B-1}\omega^{ij}\bm{e}_j\)
        \EndFor
        \State {\bf Return} \(\bm{c} = \expdVecc{\bm{c}}{B}^{\top}\)
    \end{algorithmic}
\end{algorithm}

\ifFullVersion
\begin{proof}
    Assume the recursion step gives the correct result, then \(\forall 0 \le i < B\):
    \begin{align*}
        \bm{c}_i &= \left(B r^{B-1-i}\right)^{-1} \sum_{j=0}^{B-1} \omega^{ij} \bm{e}_j \\
        &=  \left(B r^{B-1-i}\right)^{-1} \sum_{j=0}^{B-1} \left( \omega^{ij} \left(\sum_{k=0}^{B-1}r^k \omega^{kj} \bm{A}_k\right) \left(\sum_{l=0}^{B-1}r^{B-1-l} \omega^{-lj} \bm{b}_l\right)\right) \\
        &= B^{-1} \sum_{j,k,l} r^{i+k-l} \omega^{j(i+k-l)} \bm{A}_k \bm{b}_l = \sum_{k,l}r^{i+k-l} \bm{A}_k \bm{b}_l \cdot B^{-1}\sum_j \omega^{j(i+k-l)} \\
        &= \sum_{k,l} r^{i+k-l} \bm{A}_k \bm{b}_l \cdot [l \equiv i + k \pmod{B}]= \sum_{k=0}^{B-i-1} \bm{A}_k \bm{b}_{i+k} + \sum_{k=B-i}^{B-1}f \bm{A}_k \bm{b}_{i+k-B}
    \end{align*}
    The final expression is exactly the \(i\textsuperscript{th}\) block of the product \(\bm{A} \bm{b}\)

    Next we show that \(\forall 0 \le k < B\), the matrix \(\bm{M}_k\) is a \(\omega^k r\)-circulant \(\frac{N}{B} \times \frac{N}{B}\) matrix.

    Because \(\bm{A}\) is \(f\)-circulant, \(\exists (a)_{i=0}^{N-1}\) such that \(\bm{A}_{i,j} = \begin{cases}
        a_{j-i} & \text{if \(i \le j\)} \\
        f a_{N+j-i} & \text{otherwise}
    \end{cases}\). 
    
    For \(0 \le i < \frac{N}{B}\), define \(\alpha_i = \sum_{j=0}^{B-1} \omega^{jk}r^j a_{\frac{N}{B} j + i}\).

    If \(0 \le i \le j < \frac{N}{B}\):
    \[
        (\bm{M}_k)_{i,j} = \sum_{l=0}^{B-1} r^l \omega^{kl} (\bm{A}_l)_{i,j} = \sum_{l=0}^{B-1} r^l \omega^{kl} \bm{A}_{i, \frac{N}{B}l + j}
        = \sum_{l=0}^{B-1}r^l \omega^{kl} a_{\frac{N}{B}l + j - i} = \alpha_{j-i}
    \]

    If \(0 \le j < i < \frac{N}{B}\):
    \begin{align*}
        (\bm{M}_k)_{i,j} &= \sum_{l=0}^{B-1}r^l \omega^{kl} \bm{A}_{i, \frac{N}{B}l + j}= f a_{N + j - i} + \sum_{l=1}^{B-1}r^l \omega^{kl} a_{\frac{N}{B}l + j - i} \\
        &= \omega^k r \left(\sum_{l=0}^{B-2} r^l \omega^{kl} a_{\frac{N}{B} + \frac{N}{B}l + j - i}\right) + \omega^k r \left(r^{B-1} \omega^{k(B-1) a_{\frac{N}{B} + \frac{(B-1)N}{B}+j-i}}\right) \\
        &= \omega^k r \left(\alpha_{\frac{N}{B} + j - i}\right)
    \end{align*}
    Therefore \(\bm{M}_k\) is an \(\omega^k r\)-circulant matrix.

    Finally, the divide an conquer nature of the algorithm implies that the time complexity of \cref{alg:Mult2} is \(O(N \log N)\).
\end{proof}

\begin{example}
    We illustrate the recursion step of our algorithm with an example. Let \(B=3\), \(\omega\) be a primitive \(3\textsuperscript{rd}\) root of unity and \(r = \sqrt[3]{f}\) a third root of \(f\). To calculate the product
    \[\begin{pmatrix}
        \bm{A}_0 & \bm{A}_1 & \bm{A}_2 \\
        f \bm{A}_2 & \bm{A}_0 & \bm{A}_1 \\
        f \bm{A}_1 & f \bm{A_2} & \bm{A}_0
    \end{pmatrix} \begin{pmatrix}
        \bm{b}_0 \\ \bm{b}_1 \\ \bm{b}_2
    \end{pmatrix}\]
    We let
    \begin{align*}
        \bm{M}_0 &= \bm{A}_0 + r \bm{A}_1 + r^2 \bm{A}_2 \; & \bm{d}_0 &= r^2 \bm{b}_0 + r \bm{b}_1 + \bm{b}_0 \\
        \bm{M}_1 &= \bm{A}_0 + r \omega \bm{A}_1 + r^2 \omega^2 \bm{A}_2 \; & \bm{d}_1 &= r^2 \bm{b}_0 + r \omega^2 \bm{b}_1 + \omega \bm{b}_2 \\
        \bm{M}_1 &= \bm{A}_0 + r \omega^2 \bm{A}_1 + r^2 \omega \bm{A}_2 \; & \bm{d}_2 &= r^2 \bm{b}_0 + r \omega \bm{b}_1 + \bm{b}_2
    \end{align*}
    and recursively compute the product \(\bm{e}_0 = \bm{M}_0 \bm{d}_0,\, \bm{e}_1 = \bm{M}_1 \bm{d}_1,\, \bm{e}_2 = \bm{M}_2 \bm{d}_2,\). Finally, we let
    \begin{align*}
        \bm{c}_0 &= \frac{\bm{e}_0 + \bm{e}_1 + \bm{e}_2}{3r^2} \\
        \bm{c}_1 &= \frac{\bm{e}_0 + \omega \bm{e}_1 + \omega^2 \bm{e}_2}{3r} \\
        \bm{c}_1 &= \frac{\bm{e}_0 + \omega^2 \bm{e}_1 + \omega \bm{e}_2}{3}
    \end{align*}
    The final result is \(\bm{c} = (\bm{c}_0 \parallel \bm{c}_1 \parallel \bm{c}_2)^{\top}\)
\end{example}
\else
\begin{proof}
    The highlevel idea is to verify the following: 1. assume the recursion step is correct, the algorithm produces the correct output; 2. that the intermediate matrix \(\bm{M}_i\) is \(\omega^i r\)-circulant, hence the recursion step is correct. We refer the reader to \cref{proof:CircMult2} in the appendix for a detailed proof of \cref{thm:CircMult2}.
\end{proof}
\fi

\begin{remark}
    Since \(\bm{A}\) is circulant and each block \(\bm{A}_i\) in \cref{alg:Mult2} is also circulant, it is enough to store only the first row/column of the matrices. Similarly, when we compute the matrices \(\bm{M}_i\), it is sufficient to operate on the first row/column of each \(\bm{A}_i\).
\end{remark}

\subsection{Coprime Factor Case} \label{subsection:coprime}
Suppose \(N = N_1 N_2\) where \(N_1, N_2\) are relatively prime. Good-Thomas Prime Factorization \cite{JRSS:Good58,ADC:Thomas63} is a technique to reduce the computation of fourier transform of length \(N\) to the computation of 2 dimensional fourier transform of dimension \(N_1 \times N_2\). In turn its computation can be facilitated by ``nesting'' a fast algorithm for 1 dimensionaal fourier transform inside another 1 dimensional fast algorithm. This factorization can also be repeated recursively and reduce the original computation to the computation of a multi-dimensional fourier transform.

In the context of computing cyclic convolutions, we use a variant of Good-Thomas prime Factorization due to Agrawal and Cooley \cite{IEEE:AgrCoo77}. The main benefit of this approach is that it doesn't require additional roots of unity, thereby suitable to our setting where inexpensive roots of unity are scarce.

We need the notion of a stride permutation.
\begin{definition}{(Stride Permutation)}\label{def:Stride}
    Assume \(N = N_1 N_2\) where \(N_1, N_2\) are relatively prime. Let \(0 < E_1, E_2 < N\) be unique integers satisfying:
    \begin{align*}
        E_1 &\equiv 1 \pmod{N_1} \quad &E_1 &\equiv 0 \pmod{N_2} \\
        E_2 &\equiv 0 \pmod{N_1} \quad &E_2 &\equiv 1 \pmod{N_2}
    \end{align*}
    We will also call \(E_1, E_2\) the \emph{CRT basis} with respect to \(N=N_1 N_2\). Define the stride permutation associated with \(N\) as
    \[\sigma:\: \{0,\ldots, N-1\} \to \{0, \ldots, N-1\} \qquad N_2 i + j \mapsto i E_1 + j E_2 \bmod N\]
    For any \(0 \le i < N_1, 0 \le j < N_2\)
\end{definition}

\ifFullVersion
\begin{example}
    Suppose \(N = N_1 N_2 = 3 \times 4 = 12\), we can let \(E_1 = 4, E_2 = 9\). The stride permutation becomes:
    \[\begin{pmatrix}
        0 & 1 & 2 & 3 & 4 & 5 & 6 & 7 & 8 & 9 & 10 & 11\\ 
        0 & 9 & 6 & 3 & 4 & 1 & 10 & 7 & 8 & 5 & 2 & 11
    \end{pmatrix}\]
\end{example}
\fi

\begin{proposition}\label{prop:GoodThomasDecomp}
    Assume \(N = N_1 N_2\) where \(N_1, N_2\) are relatively prime. Let \(E_1, E_2\) be the CRT basis in \cref{def:Stride}. Let \(\bm{P} = \bm{P}_{\sigma}\), \(\bm{P}_{i,j} = \delta_{\sigma(i), j}\) for \(0 \le i,j < N\) be the (row)-circulant permutation matrix associated with the stride permutation in \cref{def:Stride}.
    
    Let \(\zeta_N\) be a principal \(N\textsuperscript{th}\) root of unity, and let \(\bm{V}_N\) denote the \(N \times N\) fourier matrix where \((\bm{V}_N)_{i,j} = \zeta_N^{ij}\). Then
    \begin{enumerate}
        \item If we let \(\zeta_{N_1} \coloneq \zeta_N^{E_1}\), \(\zeta_{N_2} \coloneq \zeta_N^{E_2}\). Then \(\zeta_{N_1}, \zeta_{N_2}\) are resepctively principal \(N_1\textsuperscript{th}\) and \(N_2\textsuperscript{th}\) root of unity.
        \item \(\bm{V}_N = \bm{P}^{-1} (\bm{V}_{N_1} \otimes \bm{V}_{N_2}) \bm{P}\), where \(\bm{V}_{N_1}, \bm{V}_{N_2}\) are fourier matrices associated with \(\zeta_{N_1}, \zeta_{N_2}\). \(\otimes\) denotes the matrix kronecker product.
    \end{enumerate}
\end{proposition}
\begin{proof}
    Item 1 of \cref{prop:GoodThomasDecomp} is obvious from the Chinese remainder theorem

    For the second item of \cref{prop:GoodThomasDecomp}, we let \(\sigma: \{0,\ldots,N-1\} \to \{0,\ldots,N-1\}\) be the stride permutation with respect to \(N\) defined in \cref{def:Stride}. For each \(0 \le i,j < N\) there exist unique \(0 \le a_1, a_2 < N_1\), \(0 \le b_1,b_2 < N_2\) such that
    \[i \equiv a_1 E_1 + b_1 E_2 \pmod{N} \qquad j \equiv a_2 E_1 + b_2 E_2 \pmod{N}\]
    Therefore,
    \begin{align*}
        \left(\bm{P}^{-1} (\bm{V}_{N_1} \otimes \bm{V}_{N_2}) \bm{P}\right)_{i,j} &= \left(\bm{V}_{N_1} \otimes \bm{V}_{N_2}\right)_{\sigma^{-1}(i), \sigma^{-1}(j)} = \left(\bm{V}_{N_1} \otimes \bm{V}_{N_2}\right)_{a_1 N + b1, a_2 N + b_2} \\
        &= (\bm{V}_{N_1})_{a_1, a_2} (\bm{V}_{N_2})_{b_1, b_2} = \zeta_N^{a_1 a_2 E_1 + b_1 b_2 E_2} = \zeta_N^{ij}
    \end{align*}
    Since by Chinese remainder theorem \(ij \equiv a_1 a_2 E_1 + b_1 b_2 E_2 \pmod{N}\)
\end{proof}

The following proposition describes the main idea of Agrawal Cooley algorithm. We refer the reader to \cite[Sect~7.2]{BOOK:LRTMAC89} for more detail.
\begin{proposition}\label{prop:AgrawalCooley}
    \begin{enumerate}
        \item Under the assumption of \cref{prop:GoodThomasDecomp}, the inverse fourier matrix satisfies
        \[ \bm{V}_N^{-1} = \bm{P}^{-1} \left(\bm{V}_{N_1}^{-1} \otimes \bm{V}_{N_2}^{-1}\right) \bm{P} \]
        \item Let \(\bm{u}, \bm{v}\) be 2 vectors of length \(N\). Let \(bm{H}\) be the \(N \times N\) circulant matrix whose first column is \(\bm{u}\). Let \(\bm{w} = \bm{u} \circledast \bm{v} = \bm{H} \bm{v}\) be their circular convolution product. If \(\bm{P}\) is the stride permutation matrix as in \cref{prop:GoodThomasDecomp} Then \(\bm{P} \bm{w} = \left(\bm{P} \bm{H} \bm{P}^{-1}\right) \cdot \bm{P} \bm{v}\) and \(\bm{H}_1 = \bm{P} \bm{H} \bm{P}^{-1}\) is an \(N_1 \times N_1\) block circulant matrix, each block also a circulant matrix of dimension \(N_2 \times N_2\)
        \item Hence the block fourier transform \(bm{H}_2 = \left(\bm{V}_{N_1} \otimes \bm{I}_{N_2}\right) \bm{H}_1 \left(\bm{V}_{N_1}^{-1} \otimes \bm{I}_{N_2}\right)\) is a block diagonal matrix, each block an \(N_2 \times N_2\) circulant matrix
        \item Finally, the fourier transform \(\bm{H}_3 = \left(\bm{I}_{N_1} \otimes \bm{V}_{N_2}\right) \bm{H}_2 \left(\bm{I}_{N_1} \otimes \bm{V}_{N_2}^{-1}\right)\) is a diagonal matrix
    \end{enumerate}
\end{proposition}
\begin{proof}
    Item 1 of \cref{prop:AgrawalCooley} is obtained from the well known mix product property of matrix tensor product.

    The first part of item 2 of \cref{prop:AgrawalCooley} is obvious. We now prove that \(\bm{H}_1 = \bm{P} \bm{H} \bm{P}^{-1}\) has the desired property. For \(0 \le a_1, a_2 < N_1\), \(0 \le b_1, b_2 < N_2\) we need to show that
    \[(\bm{H}_1)_{a_1 N_2 + b_1, a_2 N_2 + b_2} = (\bm{H}_1)_{0, (a_1 - a_2 \bmod N_1)N_2 + (b_2 - b_1 \bmod N_2)}\]
    But
    \begin{align*}
        &(\bm{H}_1)_{a_1 N_2 + b_1, a_2 N_2 + b_2} = \bm{H}_{a_1 E_1 + b_1 E_2 \bmod N, a_2 E_1 + b_2 E_2 \bmod N} \\
        &= \bm{u}_{(a_1 - a_2)E_1 + (b_2 - b_1)E_2 \bmod N} = (\bm{H}_1)_{0, (a_1 - a_2 \bmod N_1)N_2 + (b_2 - b_1 \bmod N_2)}
    \end{align*}
    The claim is proven.

    Item 3 and 4 of \cref{prop:AgrawalCooley} are direct consequences of the fourier transform. \(\bm{V}_{N_1} \otimes \bm{I}_{N_2}\) operates at block level and will transform a block-circulant matrix to a block-diagonal one. On the otherhand, \(\bm{I}_{N_1} \otimes \bm{V}_{N_2}\) operates parallel within each block, and will transform each circulant block into a diagonal block. 
\end{proof}

\Cref{{alg:GoodThomas}} instantiates the Good-Thomas Prime Factorization strategy. The correctness follows directly from \cref{prop:AgrawalCooley}.
\begin{algorithm}[h]
    \caption{Good-Thomas Prime Factorization}\label{alg:GoodThomas}
    \begin{algorithmic}[1]
        \Require 2 length \(N\) sequences \(\bm{u}, \bm{v}\)
        \Ensure Their circular convolution product \(\bm{w} = \bm{u} \circledast \bm{v}\)
        \State Break \(N = N_1 N_2\) into 2 coprime factors. Find the CRT basis \(E_1, E_2\) and stride permutation matrix \(\bm{P}\) as in \cref{prop:GoodThomasDecomp}.
        \State Compute permutations \(\bm{u}_1 = \bm{P} \bm{u}\), \(\bm{v}_1 = \bm{P} \bm{v}\)
        \State Calculate the block fourier transforms:
        \[ \bm{u}_2 = \left(\bm{V}_{N_1} \otimes \bm{I}_{N_2}\right) \bm{u}_1, \qquad \bm{v}_2 = \left(\bm{V}_{N_1} \otimes \bm{I}_{N_2}\right) \bm{v}_1\] 
        \Comment{This step can be recursively computed if we can break up \(N_1\) to coprime factors}
        \State Break \(\bm{u}_2 = (\bm{u}_2^{(0)} \parallel \ldots \parallel \bm{u}_2^{(N_1 - 1)})^{\top}\), \(\bm{v}_2 = (\bm{u}_v^{(0)} \parallel \ldots \parallel \bm{u}_v^{(N_1 - 1)})^{\top}\) into \(N_1\) consecutive blocks
        \For {each pair \(\bm{u}_2^{(i)}, \bm{v}_2^{(i)}\)}
            \State Calculate \(\bm{w}_2^{(i)} = \bm{u}_2^{(i)} \circledast \bm{v}_2^{(i)}\) \Comment{This step can also be recursively computed if we can break up \(N_2\) to coprime factors}
        \EndFor
        \State Let \(\bm{w}_2 = (\bm{w}_2^{(0)} \parallel \ldots \parallel \bm{w}_2^{N_1 - 1})^{\top}\). Calculate the inverse block fourier transform \(\bm{w}_1 = \left(\bm{V}_{N_1}^{-1} \otimes \bm{I}_{N_2}\right) \bm{w}_2\)
        \State {\bf Return} \(\bm{w} = \bm{P}^{-1} \bm{w}_1\)
    \end{algorithmic}
\end{algorithm}

\subsection{Combining the 2 strategies} \label{subsection:combination}
When we search for smooth divisors of \(p^r - 1\) where \(p\) is a small prime and \(r>0\) a small integer, we find that such factor \(N \mid p^r - 1\) frequently factorizes as \(N = q_1 q_2 \ldots q_{n-1} q_n^e\), \(q_1 \ldots q_n\) distinct primes. When such case occurs, it is possible to combine both strategies in \cref{subsection:coprime,subsection:primePower}. 

The highlevel idea is to first use Good-Thomas Prime Factorization and the block fourier matrices to convert the length \(N\) convolution respectively to a length \(q_1\), length \(q_1 q_2\), etc. length \(q_1 \ldots q_{n-1}\) block-wise convolution. For each block in the last step, which has size \(q_n^e \times q_n^e\), we employ \cref{alg:Mult2} to calculate the convolution. Finally, we use a series of inverse block fourier matrices associated with Good-Thomas Prime Factorization to obtain the final result.

A more precise mathematical expression of the procedure above is given by \cref{prop:TensorPrimeFactor}.
\begin{proposition} \label{prop:TensorPrimeFactor}
    Suppose \(N = q_1 \ldots q_{n-1} q_n^e\) where \(e > 0\) and \(q_1 \ldots q_n\) are distinct primes. Define
    \begin{alignat*}{5}
        &N_0 = 1, \quad &N_1 = q_1, \quad &N_2 = q_1 q_2, \quad &\dots, \quad &N_{n-1} = q_1 q_2\ldots q_{n-1} \\
        &M_0 = N, \quad &M_1 = N / q_1, \quad &M_2 = N / q_1 q_2, \quad &\dots, \quad &M_{n-1} = N / q_1 \ldots q_{n-1}
    \end{alignat*}
    For \(1 \le i \le n-1\) also let \(\bm{P}_i\) be the stride permutation matrix with respect to the factorization \(M_{i-1}=  q_i M_i\) (see \cref{def:Stride}). Define
    \begin{multline*}
        \bm{V}_{(n)} \coloneq \left(\bm{V}_{q_1} \otimes \bm{V}_{q_2} \otimes \ldots \otimes \bm{V}_{q_{n-1}} \otimes \bm{I}_{q_n^e}\right) \left(\bm{I}_{N_{n-2}} \otimes \bm{P}_{n-1}\right) \left(\bm{I}_{N_{n-3}} \otimes \bm{P}_{n-2}\right) \\ 
        \ldots \left(\bm{I}_{N_1} \otimes \bm{P}_2\right) \bm{P}_1
    \end{multline*}
    where \(\bm{V}_{q_i}\) is the fourier (Vandermonde) matrix of dimension \(q_i \times q_i\) with respect to the Good-Thomas prime factorization \(M_{i-1} = q_i M_i\) (\(q_i\) plays the role of \(N_1\) in \cref{prop:GoodThomasDecomp}).

    Let \(\bm{u}, \bm{v}\) be 2 length \(N\) sequences, \(\bm{H}\) the circulant matrix whose first column is \(\bm{u}\). Let their circular convolution \(\bm{w} = \bm{u} \circledast \bm{v} = \bm{H} \bm{v}\), then
    \[\bm{V}_{(n)} \bm{w} = \left(\bm{V}_{(n)} \bm{H} \bm{V}_{(n)}^{-1}\right) \cdot \bm{V}_{(n)}\bm{v}\]
    And \(\bm{V}_{(n)} \bm{H} \bm{V}_{(n)}^{-1}\) is a block-diagonal matrix, with \(N_{n-1} = q_1 \ldots q_{n-1}\) blocks and each block a circulant matrix of dimension \(q_n^e \times q_n^e\).
\end{proposition}
\ifFullVersion
\begin{proof}
    We prove \cref{prop:TensorPrimeFactor} by induction on the number of coprime factors \(n\) of \(N\).

    If \(n = 2,\, N = q_1 q_2^e\). Let \(N_1 = q_1\), \(N_2 = q_2^e\), \(\bm{P}_1\) the stride permutation matrix with respect to \(N = N_1 N_2\). Item 1 to 3 in \cref{prop:AgrawalCooley} implies
    \begin{align*}
        &\left(\bm{V}_{q_1} \otimes \bm{I}_{q_2^e}\right) \bm{P}_1 \bm{w} \\
        &= \left(\bm{V}_{q_1} \otimes \bm{I}_{q_2^e}\right) \left(\bm{P}_1 \bm{H} \bm{P}_1^{-1}\right) \left(\bm{V}_{q_1}^{-1} \otimes \bm{I}_{q_2^e}\right) \cdot \left(\bm{V}_{q_1} \otimes \bm{I}_{q_2^e}\right) \bm{P}_1 \bm{v} \\
        &\implies \bm{V}_{(2)} \bm{w} = \left(\bm{V}_{(2)} \bm{H} \bm{V}_{(2)}^{-1}\right) \cdot \bm{V}_{(2)} \bm{v}
    \end{align*}
    And that \(\bm{V}_{(2)} \bm{H} \bm{V}_{(2)}^{-1}\) is a block-diagnonal matrix, each block a circulant matrix of dimension \(q_2^e \times q_2^e\). This proves the base case.

    Assume the assumption holds for the number of coprime factors \(1, 2 \ldots n\). Now if \(N = q_1 q_2 \ldots q_n q_{n+1}^e\), we can also write \(N = q_1 q_2 \ldots q_{n-1} Q_n\), where \(Q_n = q_n q_{n+1}^e\). By inductive hypothesis: \(\bm{V}_{(n)} \bm{w} = \left(\bm{V}_{(n)} \bm{H} \bm{V}_{(n)}^{-1}\right) \cdot \bm{V}_{(n)} \bm{v}\), where \(\bm{V}_{(n)}\) is as in \cref{prop:TensorPrimeFactor} except we replace \(\bm{I}_{q_n^e}\) by \(\bm{I}_{Q_n}\). Again by inductive hypothesis \(\bm{V}_{(n)} \bm{H} \bm{V}_{(n)}^{-1}\) is a block-diagonal matix, each block a circulant matrix of dimension \(Q_n \times Q_n\).

    Again Item 1 to 3 in \cref{prop:AgrawalCooley} says that for \(Q_n = q_n q_{n+1}^e\), \(P_n\) the stride permutation matrix with respect to the factorization \(Q_n = q_n q_{n+1}^e\), and for any \(Q_n \times Q_n\) circulant matrix \(\bm{U}\):
    \[\left(\bm{V}_{q_n} \otimes \bm{I}_{q_{n+1}^e}\right) \left(\bm{P}_n \bm{U} \bm{P}_n^{-1}\right) \left(\bm{V}_{q_n}^{-1} \otimes \bm{I}_{q_{n+1}^e}\right)\]
    is a block-diagnoal matrix, each block is circulant of dimension \(q_{n+1}^e \times q_{n+1}^e\).

    Apply this operator to each circulant block within \(\bm{V}_{(n)}\), we obtain
    \begin{multline*}
        \left(\bm{I}_{q_1\ldots q_{n-1}} \otimes \bm{V}_{q_n} \otimes \bm{I}_{q_{n+1}^e}\right) \left(\bm{I}_{q_1\ldots q_{n-1}} \otimes \bm{P}_n\right) \left(\bm{V}_{(n)} \bm{H} \bm{V}_{(n)}^{-1}\right) \\
        \left(\bm{I}_{q_1\ldots q_{n-1}} \otimes \bm{P}_n^{-1}\right) \left(\bm{I}_{q_1\ldots q_{n-1}} \otimes \bm{V}_{q_n}^{-1} \otimes \bm{I}_{q_{n+1}^e}\right)
    \end{multline*}
    is a block diagonal matrix, each block circulant of dimension \(q_{n+1}^e \times q_{n+1}^e\). But
    \begin{align*}
        &\left(\bm{I}_{q_1\ldots q_{n-1}} \otimes \bm{V}_{q_n} \otimes \bm{I}_{q_{n+1}^e}\right) \left(\bm{I}_{q_1\ldots q_{n-1}} \otimes \bm{P}_n\right) \bm{V}_{(n)} \\
        &= \left(\bm{I}_{q_1\ldots q_{n-1}} \otimes \bm{V}_{q_n} \otimes \bm{I}_{q_{n+1}^e}\right) \left(\bm{I}_{q_1\ldots q_{n-1}} \otimes \bm{P}_n\right) \left(\bm{V}_{q_1} \otimes \bm{V}_{q_2} \otimes \ldots \otimes \bm{V}_{q_{n-1}} \otimes \bm{I}_{Q_n}\right) \\
        &\left(\bm{I}_{N_{n-2}} \otimes \bm{P}_{n-1}\right) \dots \left(\bm{I}_{N_1} \otimes \bm{P}_2\right) \bm{P}_1 \\
        &= \left(\bm{V}_{q_1} \otimes \ldots \otimes \bm{V}_{q_{n}} \otimes \bm{I}_{q_{n+1}^e}\right) \left(\bm{I}_{N_{n-1}} \otimes \bm{P}_n\right) \dots \left(\bm{I}_{N_1} \otimes \bm{P}_2\right) \bm{P}_1 \\
        &= \bm{V}_{(n+1)}
    \end{align*}
    where \(N_{n-1} = q_1 q_2 \ldots q_n\) and \(Q_n = q_n q_{n+1}^e\). In addition, we use the mix product property of tensor products to move around and combine the components in order to obtain the second last equality. This proves the induction step.
\end{proof}
\else
\begin{proof}
    The proof follows from a straightforward induction on the number of distinct prime factors \(n\) in the factorization \(N = q_1 \ldots q_{n-1} q_n^e\). We refer the reader to \cref{proof:TensorPrimeFactor} in the appendix for a detailed proof of \cref{prop:TensorPrimeFactor}.
\end{proof}
\fi

We briefly recall that the tensor product matrix \(\bm{V}_{q_1} \otimes \ldots \otimes \bm{V}_{q_{n-1}} \otimes \bm{I}_{q_n^e}\) can be decomposed into a series of block-wise operations peppered with suitable permutations. We refer the reader to \cite[Chapter~2]{BOOK:LRTMAC89} for more detail and background.

\begin{definition} \label{def:InterchangePerm}
    Let \(M, N > 0\), we define the \emph{tensor interchange permutation}
    \begin{align*}
        \tau:\: \{0, 1, \ldots, MN-1\} &\to \{0, 1, \ldots, MN- 1\} \\
        \forall 0 \le a < M, 0 \le b < N:\qquad aN + b &\mapsto bM + a
    \end{align*}
    We will let \(\bm{Q} = \bm{Q}_{\tau}\), \(\left(\bm{Q}\right)_{i,j} = \delta_{\tau(i), j}\) be the (row)-permutation matrix associated with \(\tau\).
\end{definition}

\begin{proposition} \label{prop:TensorInterchange}
    Let \(M, N > 0\), \(\bm{A}\) an \(M \times M\) matrix and \(\bm{B}\) an \(N \times N\) matrix. Using the notation in \cref{def:InterchangePerm}, we have:
    \[\bm{A} \otimes \bm{B} = \bm{Q}^{-1} \left(\bm{B} \otimes \bm{A}\right) \bm{Q}\]
\end{proposition}
\begin{proof}
    Let \(i = a_1 N + b_1\), \(j = a_2 N + b_2\) where \(0 \le a_1, a_2 < M\), \(0 \le b_1, b_2 < N\). Then
    \begin{align*}
        &\left(\bm{Q}^{-1} \left(\bm{B} \otimes \bm{A}\right) \bm{Q}\right)_{i,j} = \left(\bm{B} \otimes \bm{A}\right)_{\tau(i), \tau(j)} = \left(\bm{B} \otimes \bm{A}\right)_{b_1 M + a_1, b_2 M + a_2} = \bm{B}_{b_1, b_2} \bm{A}_{a_1, a_2} \\
        &= \left(\bm{A} \otimes \bm{B}\right)_{a_1 N + b_1, a_2 N + b_2} = \left(\bm{A} \otimes \bm{B}\right)_{i,j}
    \end{align*}
\end{proof}

\begin{corollary} \label{cor:TensorPrimeInterchange}
    Under the notation of \cref{def:InterchangePerm}, \cref{prop:TensorPrimeFactor,prop:TensorInterchange}.
    \begin{enumerate}
        \item \(\bm{V}_{q_1} \otimes \ldots \otimes \bm{V}_{q_{n-1}} \otimes \bm{I}_{q_n^e} = \prod_{i=1}^{n-1} \left(\bm{I}_{N_{i-1}} \otimes \bm{V}_{q_i} \otimes \bm{I}_{M_i}\right)\)
        \item Let \(\bm{Q}_i\) denote the tensor product interchange permutation matrix between a \(N_{i-1} \times N_{i-1}\) and a \(q_i \times q_i\) matrix (see \cref{def:InterchangePerm}). Then
            \begin{align*}
                & \bm{V}_{q_1} \otimes \ldots \otimes \bm{V}_{q_{n-1}} \otimes \bm{I}_{q_n^e}
                = \prod_{i=1}^{n-1} \left(\left(\bm{Q}_i \left(\bm{V}_{q_i} \otimes \bm{I}_{N_{i-1}}\right)\right) \otimes \bm{I}_{M_i}\right) \\
                &= \prod_{i=1}^{n-1} \left( \left(\bm{Q}_i \otimes \bm{I}_{M_i}\right) \left(\bm{V}_{q_i} \otimes \bm{I}_{N/q_i}\right) \left(\bm{Q}_i^{-1} \otimes \bm{I}_{M_i}\right)\right)
            \end{align*}
        \item Let \(\bm{Q}'_i\) denote the tensor product interchange permutation matrix between a \(q_i \times q_i\) and a \(M_i \times M_i\) matrix (see \cref{def:InterchangePerm}). Then
            \begin{align*}
                & \bm{V}_{q_1} \otimes \ldots \otimes \bm{V}_{q_{n-1}} \otimes \bm{I}_{q_n^e}
                = \prod_{i=1}^{n-1} \left( \bm{I}_{N_{i-1}} \otimes \left(\bm{Q}'_i \left(\bm{I}_{M_i} \otimes \bm{V}_{q_i}\right) \bm{Q}_i^{'-1}\right)\right) \\
                &= \prod_{i=1}^{n-1} \left(\left(\bm{I}_{N_{i-1}} \otimes \bm{Q}'_i\right) \left(\bm{I}_{N/q_i} \otimes \bm{V}_{q_i}\right) \left(\bm{I}_{N_{i-1}} \otimes \bm{Q}_i^{'-1}\right)\right)
            \end{align*}
    \end{enumerate}
\end{corollary}
\begin{proof}
    Item 1 is a direct consequence of mixed product property of tensor products. Item 2 and 3 follow from \cref{prop:TensorInterchange}.
\end{proof}

Finally, \cref{alg:PrimePowerConvolution} shows one way to instantiate the combined strategy. Correctness follows immediately from \cref{prop:TensorInterchange,cor:TensorPrimeInterchange}.
\begin{algorithm}[ht]
    \caption{Circular Convolution over prime power} \label{alg:PrimePowerConvolution}
    \begin{algorithmic}[1]
        \Require Length \(N = q_1 \ldots q_{n-1} q_n^e\), where \(q_1\ldots q_n\) are distinct small primes; 2 length \(N\) sequences \(\bm{u}, \bm{v}\); a principal \(N\textsuperscript{th}\) root of unity \(\zeta_N\)
        \Ensure Circular convolution \(\bm{w} = \bm{u} \circledast \bm{v}\)
        \State Determine the permutation matrix
        \[\bm{P} = \left(\bm{I}_{N_{n-1}} \otimes \bm{P}_{n-1}\right) \left(\bm{I}_{N_{n-3}} \otimes \bm{P}_{n-2}\right) \ldots \left(\bm{I}_{N_1} \otimes \bm{P}_2\right) \bm{P}_1\]
        where \(N_1, \ldots , N_{n-2}\) and \(\bm{P}_1, \ldots , \bm{P}_{N_{n-2}}\) are as in \cref{prop:TensorPrimeFactor}
        \State Calculate the permutations \(\bm{u}_0 = \bm{P} \bm{u}\), \(\bm{v}_0 = \bm{P} \bm{v}\)
        \For {\(i \gets 1 \text{ to } n-1\)}
            \State Permute arguments 
            \[\bm{u}_i^{(1)} = \left(\bm{Q}_i^{-1} \otimes \bm{I}_{M_i}\right) \bm{u}_{i-1}, \quad \bm{v}_i^{(1)} = \left(\bm{Q}_i^{-1} \otimes \bm{I}_{M_i}\right) \bm{v}_{i-1}\]

            where \(\bm{Q}_i\) is as in \cref{cor:TensorPrimeInterchange}
            \State Break \(\bm{u}_i^{(1)}, \bm{v}_i^{(1)}\) into \(q_i\) blocks. Compute the block fourier transform
            \[\bm{u}_i^{(2)} = \left(\bm{V}_{q_i} \otimes \bm{I}_{N / q_i}\right) \bm{u}_i^{(1)}, \quad \bm{v}_i^{(2)} = \left(\bm{V}_{q_i} \otimes \bm{I}_{N / q_i}\right) \bm{v}_i^{(1)}\]

            \(\bm{V}_{q_i}\) is obtained from \(\zeta_N\) and factorization \(M_{i-1} = q_i M_i\) as in \cref{prop:GoodThomasDecomp}
            \State Again calculate the permutations
            \[\bm{u}_i = \left(\bm{Q}_i \otimes \bm{I}_{M_i}\right) \bm{u}_i^{(2)}, \quad \bm{v}_i = \left(\bm{Q}_i \otimes \bm{I}_{M_i}\right) \bm{v}_i^{(2)}\]

            where \(\bm{Q}_i\) is as in \cref{cor:TensorPrimeInterchange}
        \EndFor
        \State Parse
        \begin{align*}
            \bm{u}_{n-1} &= (\bm{u}_{n-1, 0} \parallel \ldots \parallel \bm{u}_{n-1, N_{n-1}-1})^{\top} \\
            \bm{v}_{n-1} &= (\bm{v}_{n-1, 0} \parallel \ldots \parallel \bm{v}_{n-1, N_{n-1}-1})^{\top}
        \end{align*}
        into \(N_{n-1}\) consecutive blocks, each of size \(q_n^e\)
        \For {\(i \gets 0 \text{ to } N_{n-1} - 1\)}
            \State Let \(\bm{H}_i\) represent the circulant matrix whose first column is \(\bm{u}_{n-1, i}\)
            \State Calculate
            \[\bm{w}_{n-1, i} = \text{GenCircMatMult}(\bm{H}_i, \bm{v}_{n-1, i}, 1, q_n^e)\]

            using \cref{alg:Mult2} \Comment{\(\bm{w}_{n-1, i} = \bm{u}_{n-1, i} \circledast \bm{v}_{n-1, i}\)}
        \EndFor
        \State Combine \(\bm{w}_{n-1} = (\bm{w}_{n-1, 0} \parallel \ldots \parallel \bm{w}_{n-1, N_{n-1} - 1})^{\top}\)
        \For {\(i \gets n-1 \text{ down to } 2\)}
            \State Calculate the permutation \(\bm{w}_i^{(1)} = \left(\bm{Q}_i^{-1} \otimes \bm{I}_{M_i}\right) \bm{w}_i\), where \(\bm{Q}_i\) is as in \cref{cor:TensorPrimeInterchange}
            \State Break \(\bm{w}_i^{(1)}\) into \(q_i\) blocks. Compute the block inverse fourier transform
            \[\bm{w}_i^{(2)} = \left(\bm{V}_{q_i}^{-1} \otimes \bm{I}_{N/q_i}\right) \bm{w}_i^{(1)}\]

            \(\bm{V}_{q_i}\) is obtained from \(\zeta_N\) and factorization \(M_{i-1} = q_i M_i\) as in \cref{prop:GoodThomasDecomp}
            \State Again calculate the permutation \(\bm{w}_{i-1} = \left(\bm{Q}_i \otimes \bm{I}_{M_i}\right) \bm{w}_i^{(2)}\), where \(\bm{Q}_i\) is as in \cref{cor:TensorPrimeInterchange}
        \EndFor
        \State {\bf Return} \(\bm{w} = \bm{P}^{-1} \bm{w}_0\), where \(\bm{P}\) is as in line 1
    \end{algorithmic}
\end{algorithm}