%!TEX root = main.tex
\section{Introduction}
\paragraph{The convolution problem}
We are interested in the discrete circular convolution problem where the underlying modulus is a prime power. Specifically, given 2 sequences \(\bm{a} = \expdVec{a}{N} ,\, \bm{b} = \expdVec{b}{N} \in \Z_{p^m}^N\) for prime power modulus \(p^m\) and length \(N\). We aim to efficiently compute their circular convolution product \(\bmod \  p^m\):
\[\bm{c} = \bm{a} \circledast \bm{b} \qquad \text{where } c_i = \sum_{i=0}^{N-1}a_i b_{N-i} \, \forall 0 \le i < N\]

\ifFullVersion
Alternatively, we can recast the problem into one of wrapped polynomial multiplication. Let \(f(x) = a_0 + a_1 x + \ldots + a_{N-1}x^{N-1},\, g(x) = b_0 + b_1 x + \ldots + b_{N-1}x^{N-1} \in \Z_{p^m}[x]\), we want to efficiently calculate the product \(h(x) = f(x)g(x) \pmod{p^m, x^N - 1}\).

There is another interpretation of the problem. We define the circulant matrix \(\bm{H} \in \Z_{p^m}^{N \times N},\, \bm{H}_{i,j} = a_{j - i \bmod N}\). The goal is to efficiently calculate the matrix vector product \(\bm{c} = \bm{H} \cdot \bm{b} \in \Z_{p^m}^N\).
\fi

\paragraph{Classical Number Theoretic Transform (NTT)}
In the special(and important) case where the length \(N = 2^n\) and the modulus \(p = k2^n + 1\) is a prime, we can find a primitive \(N\textsuperscript{th}\) root of unity \(\omega \in \Z_p^{\times}\). The above problem can be handled by the famous radix-2 Cooley-Tuckey algorithm under \(O(N \log N)\) time \cite{IEEE:Cooley67}. 

\ifFullVersion
Let us briefly recall the classical NTT strategy:
\begin{enumerate}
    \item Find a multiplicative generator \(g \in \Z_p^{\times}\)
    \item Raise to a power \(\omega = g^{\frac{p-1}{N}}\) so that \(\omega\) is a primitive \(N\textsuperscript{th}\) root of unity
    \item Use Cooley-Tuckey radix 2 algorithm to compute the fourier transform \(\FFT(\bm{a})_i = \sum a_j \omega^{ij}\), \(\FFT(\bm{b})_i = \sum b_j \omega^{ij}\)
    \item Perform the elementwise product \(\bm{c}' = \FFT(\bm{a}) \odot \FFT(\bm{b})\)
    \item Finally, use the radix-2 algorithm again to calculate the inverse fourier transform \(\bm{c} = \IFFT(\bm{c}')\), \(\bm{c}_i = \frac{1}{N}\sum_{j=0}^{N-1}c'_j \omega^{-ij}\)
\end{enumerate}
\fi

\paragraph{The more general case}
Though elegant, classical NTT imposes significant restrictions on the modulus and the convolution length. Since NTT has found important applications in Cryptography, Coding Theory, Communication Theory etc. there is an ongoing research attempting to work around this limitation. The aim is to efficiently compute the circular convolution/NTT under more general modulus and length.

This paper considers this general question and focuses on prime power modulus \(p^m\) and arbitrary length. We are mostly interested in a very small prime \(p\): the most interesting case is \(p = 2\), since this is the default base in most modern computer architectures.

To the best of our knowledge, one of the most effective, practical and generic solution to the arbirary modulus, arbitrary length problem makes use of multimodular NTT, a combination of classical NTT with Residue Number System and Chinese Remainder Theorem. See for example \cite{JSC:Shoup95} Sect.6. We refer the reader to the related work section for a brief overview of this strategy.

\paragraph{Our perspectives}
In the multimodular approach, we must lift both arguments to \emph{integer sequences} and solve the integer circular convolution problem. Apart from bounding the size of the final result, the initial modulus plays almost no role in the multimodular NTT algorithm. This paper explores whether we can use NTT in a modulus-aware fashion. Namely

\begin{center}
    \it
    Is it possible to use NTT over prime power moduli \(p^m\), in such a way that the method is consistent with arithmetic \(\bmod \  p^m\)?
\end{center}

On the positive side, we will show that it is theoretically feasible to adapt NTT to handle prime power moduli and arbitrary (but very restricted) lengths. On the negative side, although our adapted NTT has a theoretical quasi-linear time complexity, a proof-of-concept implementation demonstrates that its concrete efficiency falls behind that of the multimodular NTT method and the efficiency of the state of the art modular polynomial multiplication algorithm. 

We conclude that at this stage our idea works in theory, but is not practically efficient. It is an interesting future research direction to explore whether some parts of our strategy might help improve the concrete efficiency of practical solutions to the general NTT problem.

\paragraph{Our method}
At a high level, our strategy is the same as the classical NTT and its numerous variants. Namely:
\begin{enumerate}
    \item Find a ``suitable'' set of roots of unity
    \item Pad the arguments to ``suitable'' lengths
    \item Employ ``suitable'' quasi-linear time algorithms to compute the fourier transform. pointwise multiply the intermediate result and use similar quasi-linear time algorithm to compute the inverse transform
\end{enumerate}

We need a number of less well-known techniques to make the strategy work in our setting. In more detail:
\begin{itemize}
    \item \Cref{section:roots} finds and calculates a class of roots of unity compatible with both NTT application and arithmetic of the ring \(\Z_{p^m}\). The main idea is to find these roots in the Galois ring extension \(\GalRing{p^m}{r}\). For each \(r \ge 1\), there is a suitable root of unity of order \(p^r - 1\). We will see why such a root is appropriate and provide efficient algorithms to calculate the root.
    \item In \cref{section:padding} we briefly recall how to pad arguments so that we can transform a circular convolution problem of length \(N\) to one of a more convenient length \(N' > N\).
    \item Efficient convolution and NTT is the main topic of \cref{section:ntt}. We suggest 2 algorithms based on factorization characteristics of \(N'\)
    \begin{itemize}
        \item If \(N' = q^e\) is a power of a small prime \(q\), we take inspiration from \cite{ARXIV:Rosowski21} and propose an \(O(N \log N)\) recursion algorithm to compute the circular convolution without doing the fourier and inverse fourier transforms.
        \item If \(N' = N_1 N_2\) where \(N_1, N_2\) are coprime. We employ the Good-Thomas Prime Factorization technique \cite{JRSS:Good58,ADC:Thomas63} and reduce the calculation of length \(N'\) fourier transform to 2 consecutive block-wise length \(N_1\) and length \(N_2\) fourier transforms.
        \item Finally, we combine the 2 methods above to handle those \(N'\) having factorization \(N' = q_1 \ldots q_{n-1} q_n^e\), where \(e \ge 1\) and \(q_1, \ldots, q_n\) are distinct small primes.
    \end{itemize}
    \item The last section contains some benchmark results of a proof-of-concept Sagemath/Python implementation of our strategy. We will also discuss bottlenecks and possible improvements of our approach.
\end{itemize}

\paragraph{Applications}
NTT has become a cornerstone in modern cryptographic applications, particularly within Post-Quantum Cryptography and Homomorphic Encryption. It plays a vital role in lattice-based cryptography schemes such as Dilithium, Falcon, and Kyber \cite{IACR:Dilithium17,ASIACRYPT:Falcon18,IEEE:Kyber18}.  In Homomorphic Encryption schemes like BFV, BGV, and CKKS \cite{ACM:Brakerski12,IEEE:BGV11,ASIACRYPT:CKKS17}, NTT is also critical for such operations as modulus switching and ciphertext relinearization. However as hinted before, classical NTT imposes severe restrictions on the set of parameters, particularly the existence of suitable \(N\textsuperscript{th}\) or \(2N\textsuperscript{th}\) roots of unity (for negacyclic convolution) modulo a prime \(q\). We believe that overcoming the limitation of classical NTT will also open up the range of parameter choices in those schemes and will help us find better and more secure instantiations.

If we focus on the power of 2 modulus, where \(p = 2\), our perspective might also have implications in domains such as Coding Theory and Communication. Efficient computation of (linear) circular convolutions is integral to the encoding and decoding procedures of certain codes (see for example \cite{UNC:Berlekamp69,AMC:DouSah22} for cyclic and negacyclic codes). In addition, our result might also testify to the perspective that some operations on finite fields have useful analogues in finite ring settings.