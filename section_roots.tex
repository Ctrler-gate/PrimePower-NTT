%!TEX root = main.tex
\section{On roots of unity}
Whether NTT or FFT, all fourier transform related algorithms require certain roots of unity. Suppose we need to compute the fourier transform of \(N\) data points, we generally need \(N\) distinct roots of unity, but the nature of these roots depends on context.

In scientific computations, data are real/complex numbers in general. The \(N\) roots of unity are usually taken to be the evenly-spaced points on the complex unit circle \(0 \le k < N:\: \zeta^k \coloneq \exp{(\frac{2 \pi i k}{N})}\).

In discrete computational problems, data are usually finite field/finite domain elements. The often repeated mantra is that the \(N\) roots of unity are generated by a \emph{primitive} \(N\textsuperscript{th}\) root of unity, an element \(\zeta\) in the field/domain such that \(\zeta^N = 1\), \(\forall 0 < k < N:\: \zeta^k \neq 1\).

In non-integral rings, for example \(\Z_{32}, \Z_{24}\), primitivity alone no longer suffices
\begin{example}
    \(x^2 - 1 = 0\) has 4 solutions over \(\Z_8\): \(x = 1, 3, 5, 7 \pmod{8}\), among which \(3, 5, 7\) are all primitive \(2\textsuperscript{nd}\) roots of unity. Are they suitable for a length-2 NTT? Let us look at the the fourier(Vandermonde) matrix of the root \(\zeta = 5\).
    \[\bm{V} = \begin{pmatrix}
        \zeta^{0} & \zeta^{0} \\
        \zeta^{0} & \zeta^{1}
    \end{pmatrix} = \begin{pmatrix}
        1 & 1 \\
        1 & 5
    \end{pmatrix} \qquad \det{(\bm{V})} = 5 - 1 = 4\]
    \(\bm{V}\) is not even inverible over \(\Z_8\). This is also true for other primitive roots of unity.
\end{example}

On close inspection, the non-invertibility of \(\bm{V}\) is the only obstruction. We conclude that in the context of non-integral rings, the \(N\textsuperscript{th}\) root of unity suitable for NTT/FFT application is one for which its corresponding fourier matrix \(\bm{V}\) is invertible.

\begin{definition}{(Principal root of unity, adapted from~\cite{MISC:WikiRoot})} \label{def:PrincipalRoot}
    Let \(\RingR\) be a commutative ring with identity. We call \(\zeta \in \RingR\) a principal \(N\textsuperscript{th}\) root of unity if
    \begin{enumerate}
        \item \(N\) is relatively prime with \(\mathsf{char}(\RingR)\)
        \item \(\zeta^N = 1\)
        \item \(\forall 1 \le k < N:\: \sum_{i=0}^{N-1}\zeta^{ik} = 0\)
    \end{enumerate}
\end{definition}

The following proposition is a direct consequence of \cref{def:PrincipalRoot}.
\begin{proposition} \label{prop:InvFourierMat}
    If \(\zeta\) is a principal \(N\textsuperscript{th}\) root of unity, then the fourier matrix \(\bm{V} \coloneq \bm{V}(1, \zeta, \ldots, \zeta^{N-1})\), \(\forall 0 \le i,j < N:\:\bm{V}_{i,j} = \zeta^{ij}\) is invertible with inverse \(\bm{V}^{-1} = \frac{1}{N}\bm{V}^{\ast}\), \(\bm{V}^{\ast}_{i,j} = \zeta^{-ij}\).
\end{proposition}

\ifFullVersion
Let \(p \ne 2\) be a prime. A well known result due to Gau{\ss} says for any \(m \ge 1\), the unit group \(\Z_{p^m}^{\times} \cong \Cyclic{\phi(p^m)} = \Cyclic{p^{m-1}(p-1)}\) is cyclic. In fact, all units in the (unique) order \((p-1)\) subgroup of \(\Z_{p^m}^{\times}\) are principal.
\begin{proposition} \label{prop:SubgrpPrincipal}
    Let \(\zeta \in \Z_{p^m}^{\times}\) generates the unique subgroup of order \((p - 1)\). Then \(\zeta\) is a principal \((p-1)\textsuperscript{th}\) root of unity.
\end{proposition}
\begin{proof}
    Let \(N = p - 1\). 1 and 2 in \cref{def:PrincipalRoot} is obvious. Note that for any \(1 \le k < N\), \((\zeta^k - 1)(1 + \zeta^k + \ldots + \zeta^{k(N-1)}) = \zeta^{Nk} - 1 = 0\). Since \(\zeta \bmod p\) is a primitive \((p-1)\textsuperscript{th}\) root of unity in \(\Z_p\), \(\zeta^k - 1 \ne 0 \pmod{p}\). Hence \(\zeta^k - 1\) is relatively prime to \(p^m\) and is invertible over \(\Z_{p^m}\). This proves 3 in \cref{def:PrincipalRoot}.
\end{proof}
\fi

Although \(\Z_{p^m}\) contains a principal \((p-1)\textsuperscript{th}\) root of unity, which is a generator of the unique cyclic subgroup of \(\Z_{p^m}^{\times}\) order \((p - 1)\), the problem is that in most cases of interest, the convolution length \(N \gg p\) (this holds in particular when \(p = 2\)). To find large principal roots of unity while still preserving the modular structure forces us to look at extension rings. This is where the Galois Ring comes into our picture.

Galois Rings can be motivated, defined and represented in a number of different ways. We refer the reader to \ToRef{Galois Rings} for more backgrounds and theories. Here we would like to think of a Galois Ring \(\GalRing{p^m}{r}\) as degree \(r\) extension of \(\Z_{p^m}\) in the same way that the Galois field \(\mathbb{F}_{p^r}\) is a degree \(r\) extension of \(\Z_p\).

\begin{definition}{(Galois Ring~\cite{MISC:WikiGalRing})} \label{def:GaloisRing}
    The Galois Ring \(\GalRing{p^m}{r}\) can be represented by a quotient polynomial ring \(\Z[x] / (p^m, f(x))\) where \(f(x)\) is a degree \(r\) monic polynomial which is also irreducible \(\bmod \  p\).
\end{definition}

Just like finite fields, all Galois Rings with the same modulus \(p^m\) and extension degree \(r\) are isomorphic. In some sense \(\GalRing{p^m}{r}\) doesn't depend on the particular choice of \(f(x)\). However, some \(f(x)\) are more convienient from a computational point of view.

In finite field theory, a degree \(r\) polynomial \(f(x) \in \Z_p[x]\) is called \emph{primitive} if \(f(x)\) is monic irreducible and \(x \pmod{p, f(x)}\) has order \(p^r - 1\) in \(\mathbb{F}_{p^r} \cong \Z_p[x]/(f(x))\). In other words, the equivalent class of \([x]\) is a primitive \((p^r - 1)\textsuperscript{th}\) root of unity.

We would like to find analogues of primitive polynomials over the ring \(\Z_{p^m}[x]\). Indeed, using Hensel's lifting technique, we can lift a primitive polynomial \(f(x) \in \Z_p[x]\) to \(F(x) \in \Z_{p^m}[x]\). We will show that the lifted polynomial has some desirable properties.
\begin{theorem}{(Hensel Lifting, integral form~\cite{MISC:WikiHensel})} \label{thm:HenselLifting}
    Suppose \(f(x) \equiv \alpha_0 g(x) h(x) \pmod{p}\), where \(\alpha_0\) is not divisible by \(p\) and \(g(x), h(x)\) are monic polynomials that are coprime \(\bmod \  p\). Then \(\forall k \ge 1\) there exist polynomials \(g_k(x), h_k(x) \in \Z[x]\) unique up to \(\bmod \  p^k\) such that
    \begin{enumerate}
        \item \(g_k(x) \equiv g(x) \pmod{p} \qquad f_k(x) \equiv f(x) \pmod{p}\)
        \item \(f(x) \equiv \alpha_0 g_k(x) f_k(x) \pmod{p^k}\)
    \end{enumerate}
\end{theorem}
\begin{proposition} \label{prop:LiftedPrimPoly}
    Let \(f(x)\) be a primitive polynomial \(\bmod \  p\) of degree \(r\), there exists a monic polynomial \(g(x)\) unique up to \(\bmod \  p\) such that
    \[x^{p^r - 1} - 1 \equiv f(x) g(x) \pmod{p}\]

    Now apply \cref{thm:HenselLifting} Hensel lifting to the equation above. We can find a unique polynomial \(f_m(x) \in \Z_{p^m}[x]\) such that \(f_m(x) \equiv f(x) \pmod{p}\) and \(f_m(x) \mid x^{p^r - 1} - 1\) over \(\Z_{p^m}[x]\).

    Let the Galois Ring be defined over this polynomial \(\GalRing{p^m}{r} \cong \Z[x] / (p^m, f_m(x))\). We claim that:
    \begin{enumerate}
        \item The equivalent class \(x \pmod{p^m, f_m(x)}\) has order \(p^r - 1\) over \(\GalRing{p^m}{r}^{\times}\). Moreover,
        \item The equivalent class \(x \pmod{p^m, f_m(x)}\) is a principal \((p^r - 1)\textsuperscript{th}\) root of unity over \(\GalRing{p^m}{r}\). Hence
        \item For any \(N \mid p^r - 1\), the equivalent class \(x^{\frac{p^r - 1}{N}} \pmod{p^m, f_m(x)}\) is a principal \(N\textsuperscript{th}\) root of unity.
    \end{enumerate}
\end{proposition}
\begin{proof}
    \noindent1. \quad Since \(f_m(x) \mid x^{p^r -  1} - 1\) over \(\Z_{p^m}[x]\), \(x^{p^r - 1} \equiv 1 \pmod{p^m, f_m(x)}\). Suppose there exists a \(0 < k < p^r - 1\) such that \(x^k \equiv 1 \pmod{p^m, f_m(x)}\). Since \(f_m(x) \equiv f(x) \pmod{p}\), we can reduce \(\bmod \  p\) and obtain \(x^k \equiv 1 \pmod{p, f(x)}\), contradiction to the fact that \(f(x)\) is a primitive polynomial \(\bmod \  p\).

    \smallskip
    \noindent2. \quad The only nontrivial part to verify is condition 3 in \cref{def:PrincipalRoot}. Let \(N = p^r - 1\) and fix \(0 < k < N\). It is easy to see that \((x^k - 1)(\sum_{i=0}^{N-1} x^{ik}) = x^{kN} - 1 \equiv 0 \pmod{p^m, f_m(x)}\). We claim that \(x^k - 1 \pmod{p^m, f_m(x)}\) has a multiplicative inverse over \(\Z_{p^m}[x]\). If the claim is true, we can multiply the inverse and obtain \(\sum x^{ik} \equiv 0 \pmod{p^m, f_m(x)}\). The equivalent class \([x]\) is therefore a principal \((p^r - 1)\textsuperscript{th}\) root of unity.

    \medskip
    \noindent{\bf Claim:} \(\exists h(x):\: (x^k - 1)h(x) \equiv 1 \pmod{p^m, f_m(x)}\)

    \noindent\emph{proof of Claim.} \quad Since \(0 < k < N\), \(x^k - 1 \pmod{p, f(x)}\) is nonzero and has a multiplicative inverse \(h_1(x)\), i.e., \((x^k - 1)h_1(x) \equiv 1 \pmod{p, f_m(x)}\) (recall that \(\Z[x]/(p, f(x)) = \Z[x]/(p, f_m(x))\) is a field). We are going to lift the inverse \(\bmod \  p\) to one \(\bmod \  p^m\). To do so it is most convenient to employ a technique known as Newton-Raphson division~\cite{MISC:WikiNewton}

    \smallskip
    \noindent{\bf Newton-Rhaphson division: }Let \(l > 0\). Suppose \(\exists h_l(x)\) s.t. \((x^k - 1)h_l(x) \equiv 1 \pmod{p^l, f_m(x)}\). Define
    \[h_{l+1}(x) = 2 h_l(x) - (x^k - 1)h_l(x)^2\]
    Then \((x^k - 1)h_{l+1}(x) \equiv 1 \pmod{p^{l+1}, f_m(x)}\)

    \noindent\emph{proof of lifting.} \quad Write \((x^k - 1)h_l(x) = 1 = p^l M_l(x) + f_m(x) N_l(x)\) for some polynomials \(M_l(x), N_l(x)\). Then
    \begin{align*}
        &(x^k - 1)^2h_l(x)^2 = 1 + 2p^l M_l(x) + p^{l+1}(p^{l-1}M_l(x)^2) + f_m(x)\left(f_m(x) N_l(x)^2 + 2 N_l(x) (1 + p^l M_l(x)) \right) \\
        &2(x^k - 1)h_l(x) = 2 + 2p^l M_l(x) + f_m(x) N_l(x) \\
        \implies &(x^k - 1)h_{l+1}(x) = 1 + p^{l+1} M_{l+1}(x) + f_m(x) N_{l+1}(x)
    \end{align*}
    for some polynomials \(M_{l+1}(x), N_{l+1}(x)\). Therefore we can use Newton-Rhaphson division to lift the inverse up to \(\bmod \  p^m\). This shows that \(x^k - 1\) and the claim is proven.

    \medskip
    \noindent3. \quad is a straightforward consequence of 2.
\end{proof}